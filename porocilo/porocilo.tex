\documentclass[11pt,a4paper]{article}

\usepackage[slovene]{babel}
\usepackage[utf8x]{inputenc}
\usepackage{graphicx}
\usepackage{hyperref}
\usepackage{rotating}
\usepackage{adjustbox}

\pagestyle{plain}

\begin{document}
\title{Poročilo pri predmetu \\
Analiza podatkov s programom R}
\author{Samo Metličar}
\maketitle

\section{Izbira teme}

Za temo sem izbral McDonald's, obravnaval pa bom njihovo ponudbo, razširjenost ter delovanje podjetja samega.

\section{Obdelava, uvoz in čiščenje podatkov}

Iz \href{http://nutrition.mcdonalds.com/getnutrition/nutritionfacts.pdf}{McDonalds Nutrition Facts} sem dobil prvo tabelo, ki vsebuje hranilne vrednosti njihovih jedi. Datoteka je bila v PDF formatu ter večina imen je bila več-vrstičnih, kar je povzročalu probleme tako Excelu kot Wordu. Ko sem prenesel vse podatke v Excel, sem shranil tabelo kot CSV datoteko v \verb|podatki/nutrition.csv|. Nato sem se lotil druge tabele, ki sem jo pridobil iz \href{http://nutrition.mcdonalds.com/getnutrition/ingredientslist.pdf}{McDonalds Ingredient List}. Ta tabela mi je povzročala malce več težav in po predlogu asistenta sem se odločil, da jo bom pretvoril v \verb|TRUE/FALSE| tabelo, kar sem storil s pomocjo R. Ker pa se vecina jedi ne pojavla pogosto, sem moral močno zožiti izbor jedi, zato je tabela manjših razsegov. \par

Obe tabeli sem uvozil s funkcijama v \verb|uvoz/uvoz.r| ter ju na to s funkcijami v \verb|lib/tabeli.r| še dopolnil ter pretvoril v željeni tabeli. Sledilo je ustvarjanje grafov s pomočjo funkcij v \verb|lib/graf1.r|, ki izriše vse tri grafe v PDF dokument \verb|slike/grafi.pdf|. \par

Na koncu je sledil še uvoz podatkov iz Wikipedije, kjer sem dobil podatke o državah, kjer McDonalds obratuje itd. Po uvozu teh podatkov sem še nekoliko uredil tabelo ter spremenil nekatere vrednosti. Na koncu pa sem še pretvoril datume iz niza v \verb|Date| ter jih izpisal v nam bolj prijazni oblik.

\includegraphics[width=\textwidth]{../slike/graf1.pdf}
\includegraphics[width=\textwidth]{../slike/graf2.pdf}
\includegraphics[width=\textwidth]{../slike/graf3.pdf}

\newpage

\section{Analiza in vizualizacija podatkov}


V tretji fazi sem uvozil zemljevid sveta s spletne strani \href{http://www.naturalearthdata.com/http//www.naturalearthdata.com/download/50m/cultural/ne_50m_admin_0_countries.zip}{Natural Earth Data}, ki jo je svetoval asistent ter se lotil spreminjanja tabel, ki so mi bile na voljo, da so ustrezale le-temu zemljevidu. Uvozni program najdemo v \verb|lib/uvozi.zemljevid.r|. \par

Nato pa sem spremenil ter dodelal program \verb|vizualizacija/vizualizacija.r|, ki shrani zemljevid s spleta ter ga preuredi v dva zemljevida. Prvi prikazuje, koliko McDonaldsov je v posameznih državah po svetu, drugi pa kdaj je bila odprta prva restavracija v tisti državi.

Prvo sem se lotil števila restavracij po državah, tako da sem popravil ter uskladil imena z zemljevida ter mojih podatkov, da sta se ujemala v številu držav, kot tudi katere sta vsebovala. Nato pa sem izrisal graf s funkcijo \verb|spplot|. \par

Za drug zemljevid sem podatke uredil po datumih ter jim po desetletjih dodelil barve, po letnicah znotraj desetletja pa \verb|transparency|. Nato pa sem želel dodati še imena nekaterih največjih držav, kjer McDonalds posluje ter jih tudi označiti, zato sem na spletu poiskal \href{https://developers.google.com/public-data/docs/canonical/countries_csv}{koordinate središč držav} ter \href{http://simple.wikipedia.org/wiki/List_of_countries_by_area}{njihove površine}. Oba seznama sem pretvoril ter naložil, kot \verb|.CSV| datoteki ter ju uvozil v moj program. Podatke sem uredil, tako da so se ujemali s seznamom držav iz zemljevida ter prejšnjimi podatki ter ustvaril seznam imen ter koordinat, ki sem jih želel izpisati. Za največje države sem izpisal celotno ime, medtem ko sem za manjše izpisal kratico ali pa jo samo označil s krogcem. Zemljevid sem nato izrisal v \verb|.PDF| obliki. \par

V tej fazi mi povzroča probleme predvsem \verb|spplot|, ki ga ne želi zapisati v \verb|.PDF|, razen če se še posebaj zaženejo tiste vrstice. \newpage

\begin{sidewaysfigure}[ht]
	\includegraphics[width=\textwidth]{../slike/zemljevid1.pdf}
\end{sidewaysfigure}


\begin{sidewaysfigure}[ht]
	\includegraphics[width=\textwidth]{../slike/zemljevid2.pdf}
\end{sidewaysfigure}

\newpage

\section{Napredna analiza podatkov}



\end{document}
